\documentclass[a4paper, 10pt]{jsarticle}
\usepackage[utf8]{inputenc}
\usepackage[T1]{fontenc}
\usepackage{geometry}
\geometry{left=25mm, right=25mm, top=25mm, bottom=25mm}
\usepackage{hyperref}
\usepackage[dvipdfmx]{graphicx}
\usepackage[dvipdfmx]{xcolor}
\usepackage[dvipdfmx]{hyperref}
\usepackage{listings}
\usepackage{adjustbox}


\title{Webプログラミング仕様書}
\author{25G1137 吉田朝朗}
\date{\today}

\begin{document}

\maketitle

\begin{center}
    \vspace{-10pt} 
    \textbf{【ソースコード・リポジトリ】}\par
    \vspace{5pt}
    \url{https://github.com/tomoro1729-a/webpro_06.git}
    \vspace{20pt} 
\end{center}



\section{開発者向け仕様書}
本システムは,Node.jsおよびExpressフレームワークを用いて構築された,3つの異なるテーマ(映画・化学元素・ゲーム武器)を扱うWebデータベースアプリケーションである.各システムは独立したリソースとして管理されつつも,統一されたRESTfulな設計指針に基づいて実装されている.

\subsection{技術スタックと構成}
\begin{itemize}
    \item \textbf{サーバーサイド}: Node.js (Express)
    \item \textbf{テンプレートエンジン}: EJS (HTML動的生成)
    \item \textbf{フロントエンド}: 標準HTML5, CSS3(静的ファイルとして外部定義)
    \item \textbf{データ管理}: サーバー内メモリによるオブジェクト配列管理
\end{itemize}

\section{共通設計仕様}

\subsection{ページ構造の決定}
各サブシステムは,ユーザーの操作性を考慮し,以下の4種類のページ構造を共通で持つ.
\begin{enumerate}
    \item \textbf{一覧ページ (Index)}: 登録データの全件をテーブル形式で表示する.
    \item \textbf{詳細ページ (Show)}: 特定リソースの全プロパティを閲覧する.
    \item \textbf{新規登録ページ (New)}: 新規データ入力を受け付けるフォーム.
    \item \textbf{編集ページ (Edit)}: 既存データの修正を受け付けるフォーム.
\end{enumerate}

\subsection{ページ遷移と処理の流れ}
本システムはステートレスなHTTP通信を行うが,操作後のユーザー体験(UX)を最適化するため,以下のリダイレクト戦略を採用している.



\begin{itemize}
    \item \textbf{新規作成/削除後}: データの増減を即座に確認させるため,「一覧ページ」へリダイレクトする.
    \item \textbf{編集完了後}: 修正された内容を再確認させるため,「詳細ページ」へリダイレクトする.
\end{itemize}

\section{各サブシステムの詳細設計}

\subsection{システム1:ドラえもん映画一覧システム}
歴代作品の興行データ等を管理する.

\subsubsection{データ構造}
\begin{table}[h]
\centering
\begin{adjustbox}{width=\textwidth}
\begin{tabular}{|l|l|l|p{7cm}|}
\hline
\textbf{プロパティ} & \textbf{型} & \textbf{必須} & \textbf{内容} \\ \hline
id & Integer & 〇 & システム一意識別子 \\ \hline
title & String & 〇 & 映画の題名 \\ \hline
year & Integer & 〇 & 公開された西暦 \\ \hline
revenue & String & - & 興行収入記録 \\ \hline
explain & String & - & あらすじ・概要説明 \\ \hline
\end{tabular}
\end{adjustbox}
\end{table}

\subsection{システム2:周期表元素一覧システム}
原子番号を一意識別子(ID)として利用し,科学データを管理する.

\subsubsection{データ構造}
\begin{table}[h]
\centering
\begin{adjustbox}{width=\textwidth}
\begin{tabular}{|l|l|l|p{7cm}|}
\hline
\textbf{プロパティ} & \textbf{型} & \textbf{必須} & \textbf{内容} \\ \hline
number & Integer & 〇 & 原子番号(ID兼用) \\ \hline
symbol & String & 〇 & 元素記号(H, He等) \\ \hline
name & String & 〇 & 日本語名 \\ \hline
explain & String & - & 性質や発見の経緯等の記述 \\ \hline
\end{tabular}
\end{adjustbox}
\end{table}

\subsection{システム3:FE風花雪月 武器一覧システム}
ゲームバランスに関わる複数の数値パラメータを管理する.

\subsubsection{データ構造}
\begin{table}[h]
\centering
\begin{adjustbox}{width=\textwidth}
\begin{tabular}{|l|l|l|p{7cm}|}
\hline
\textbf{プロパティ} & \textbf{型} & \textbf{必須} & \textbf{内容} \\ \hline
id & Integer & 〇 & 武器一意識別子 \\ \hline
name & String & 〇 & 武器の名称 \\ \hline
type & String & 〇 & 分類(剣,槍,斧,弓等) \\ \hline
might / hit & Integer & - & 威力および命中率 \\ \hline
explain & String & - & 武器の特性や特殊効果の記述 \\ \hline
\end{tabular}
\end{adjustbox}
\end{table}

\section{APIエンドポイント設計(共通)}
本システムは,リソース名(\texttt{/doraemon}, \texttt{/element}, \texttt{/weapon})に対して以下の統一的なパス構造を適用している.

\begin{table}[h]
\centering
\begin{tabular}{|l|l|l|l|}
\hline
\textbf{機能} & \textbf{メソッド} & \textbf{パス(例)} & \textbf{レスポンス} \\ \hline
一覧表示 & GET & \texttt{/resource} & 一覧画面のレンダリング \\ \hline
詳細表示 & GET & \texttt{/resource/:id} & 詳細画面のレンダリング \\ \hline
新規登録実行 & POST & \texttt{/resource/create} & 一覧へリダイレクト \\ \hline
更新実行 & POST & \texttt{/resource/update/:id} & 詳細へリダイレクト \\ \hline
削除実行 & POST & \texttt{/resource/delete/:id} & 一覧へリダイレクト \\ \hline
\end{tabular}
\end{table}

\section{実装上の工夫と安全性}
\begin{itemize}
    \item \textbf{一貫性の維持}: CSSファイルを \texttt{/public/style.css} に外部化し,全リソース間でデザインの整合性を確保した.
    \item \textbf{入力制御}: フォームの数値項目には \texttt{type="number"} を使用し,ブラウザ側でのバリデーション機能を活用した.
    \item \textbf{削除の安全化}: 削除操作は \texttt{POST} メソッドによる送信と確認ダイアログ(JavaScript)を併用し,意図しないデータ消失を防止した.
\end{itemize}

% =================================================================
% 管理者向けマニュアルセクション
% =================================================================

\newpage
\section{管理者向け仕様書}

本章では,本システムの導入(インストール)から,日常的な起動・停止手順,およびトラブルシューティングについて記述する.

\subsection{ソースコードの入手と環境構築}

\subsubsection{動作要件}
本システムを稼働させるためには,以下のソフトウェア環境が必要である.
\begin{itemize}
    \item \textbf{OS}: Windows, macOS, Linux いずれか
    \item \textbf{Runtime}: Node.js (LTS推奨)
    \item \textbf{Manager}: npm (Node.jsに標準同梱)
\end{itemize}

\subsubsection{インストール手順}
以下の手順に従い,GitHubリポジトリからソースコードを取得し,依存ライブラリをインストールする.

\begin{enumerate}
    \item \textbf{リポジトリのクローン}:
    ターミナル(コマンドプロンプト)を開き,以下のコマンドを実行してソースコードをダウンロードする.
\begin{lstlisting}[language=bash, caption=Gitクローンコマンド]
git clone https://github.com/tomoro1729-a/webpro_06.git
cd webpro_06
\end{lstlisting}

    \item \textbf{依存パッケージのインストール}:
    プロジェクトフォルダ内で以下のコマンドを実行し,\texttt{package.json} に記載されたライブラリ(Express, EJS等)を一括導入する.
\begin{lstlisting}[language=bash, caption=パッケージインストールコマンド]
npm install
\end{lstlisting}
\end{enumerate}

\subsection{システムの起動と終了}

\subsubsection{起動方法}
環境構築完了後,以下のコマンドでWebサーバーを起動する.
\begin{lstlisting}[language=bash]
node app.js
\end{lstlisting}

起動に成功すると,コンソールに \texttt{Server is running on http://localhost:8080} と表示される.
ブラウザで上記URLへアクセスすることでシステムを利用開始できる.

\subsubsection{終了方法}
サーバーを停止する場合は,ターミナル上でキーボードの \textbf{Ctrl + C} を入力する.これによりプロセスが安全に終了する.

\subsection{トラブルシューティング}
起動時または操作時に問題が発生した場合の対処法を以下に示す.

\begin{table}[h]
\centering
\caption{よくあるエラーと対処法}
\begin{adjustbox}{width=\textwidth}
\begin{tabular}{|l|p{6cm}|p{7cm}|}
\hline
\textbf{現象・エラーメッセージ} & \textbf{推定される原因} & \textbf{対処法} \\ \hline
\texttt{command not found: node} & Node.jsがインストールされていない,またはパスが通っていない. & 公式サイトよりNode.jsのインストーラをダウンロードし,セットアップを行う. \\ \hline
\texttt{Error: Cannot find module...} & 依存ライブラリ(\texttt{node\_modules})が不足している. & \texttt{npm install} コマンドを再度実行し,完了を待つ. \\ \hline
\texttt{EADDRINUSE :::8080} & ポート8080番が既に使用されている. & 既に起動している別の\texttt{node}プロセスを終了するか,PCを再起動する. \\ \hline
\end{tabular}
\end{adjustbox}
\end{table}

\subsection{システムの制約事項と既知の課題}
本システムは学習用プロトタイプであり,運用上以下の制約が存在する.

\begin{itemize}
    \item \textbf{データの揮発性}:
    データベースを使用せず,サーバーのメモリ上(変数)でデータを管理している.そのため,\textbf{サーバーを再起動・停止すると,追加・編集したデータは初期状態にリセットされる}.恒久的な運用を行う場合は,別途データベース(SQLite, MySQL等)の実装が必要である.
    
    \item \textbf{同時編集の競合}:
    排他制御を行っていないため,複数の管理者が同一データを同時に編集した場合,最後に保存操作を行った内容で上書きされる仕様である.
\end{itemize}

% =================================================================
% 利用者向けマニュアルセクション
% =================================================================

\newpage
\section{利用者向け仕様書}

本章では,一般利用者を対象に,本データベースシステムの基本的な操作手順について解説する.
本システムは「ドラえもん映画」「元素周期表」「武器データ」の3つのサブシステムから構成されているが,すべての画面で共通の操作方法を採用している.ここでは例として,システムごとの共通手順を示す.

\subsection{システムへのアクセス}

\subsubsection{トップメニュー}
ブラウザでシステムにアクセスすると,最初にメニュー画面が表示される.
利用したいデータベースの「システムを開く」ボタンをクリックすることで,各管理画面へ移動する.

% 画像挿入エリア(画像ファイルがない場合はコンパイルエラーになるため、画像を配置するかコメントアウトしてください)
\begin{figure}[h]
    \centering
    % \includegraphics[width=10cm]{menu.png} 
    % ※画像を保存したら上の行の % を外してください
    \caption{システム選択メニュー画面}
\end{figure}


\subsection{データの閲覧}

\subsubsection{一覧画面(リスト表示)}
各システムのトップ画面では,登録されている全データが表形式で一覧表示される.
\begin{itemize}
    \item \textbf{詳細リンク}: データのIDや名前をクリックすると,そのデータの詳細情報画面へ移動する.
    \item \textbf{削除ボタン}: データの右側にある削除ボタンを押すと,その行のデータが削除される(詳細は後述).
\end{itemize}

\begin{figure}[h]
    \centering
    % サイズを10cmから7cmに変更しました
    \includegraphics[width=10cm]{list.png}
    \caption{データ一覧表示画面の例}
\end{figure}

\subsubsection{詳細画面}
一覧画面から選択したデータの全項目(説明文などの長いテキスト含む)を確認できる.
この画面には「編集」ボタンがあり,登録内容の修正を行うことができる.

\begin{figure}[h]
    \centering
    % サイズを10cmから7cmに変更しました
    \includegraphics[width=10cm]{detail.png}
    \caption{詳細情報確認画面}
\end{figure}


\subsection{データの管理(追加・編集・削除)}

\subsubsection{新規データの追加}
一覧画面にある「新規登録(Create New)」リンクをクリックすると,登録フォームが表示される.
必要な項目(名前,数値など)を入力し,「保存」ボタンを押すことで新しいデータがデータベースに追加される.

\begin{figure}[h]
    \centering
    % サイズを10cmから7cmに変更しました
    \includegraphics[width=10cm]{form.png}
    \caption{データ入力フォーム}
\end{figure}

\subsubsection{既存データの編集}
詳細画面にある「編集(Edit)」ボタンをクリックすると,現在のデータが入力された状態でフォームが開く.
内容を修正して「更新」ボタンを押すと,変更内容が反映される.

\subsubsection{データの削除}
一覧画面の各行にある「削除(Delete)」ボタンをクリックする.
\textbf{注意}: 削除操作を行うとデータは即座に消去され,元に戻すことはできないため,慎重に操作を行うこと.

\end{document}